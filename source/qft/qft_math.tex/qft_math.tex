\documentclass[a4paper]{article} 

\def\ncourse {Quantum Field Theory}
\def\ntopic {1 - mathematical formalism of the S-matrix}
 
\makeatletter

%Title
\def\nauthor{Matteo Zandi}

\title{\huge \ncourse \\ \Large \ntopic}

\author{\color{mycolor}\nauthor}
\date{\today}

%Size
\usepackage{geometry}
%\geometry{a4paper, top = 4cm, bottom = 4cm, left = 3cm, right = 4cm}
\geometry{
    papersize={379pt, 699pt},
    textwidth=345pt,
    textheight=596pt,
    left=17pt,
    top=54pt,
    right=17pt
  }

%Package
\usepackage{lipsum}
\usepackage{xcolor} \xdefinecolor{mycolor}{RGB}{0,175,179} 
\usepackage{hyperref} \hypersetup{colorlinks, linkcolor={mycolor}, citecolor={mycolor}, urlcolor={mycolor}}
\usepackage{titlesec}
\titleformat{\section}{\newpage\normalfont\Large\bfseries\color{mycolor}\centering}{\thesection}{1em}{}
\titleformat{\subsection}{\normalfont\large\bfseries\color{mycolor}\centering}{\thesubsection}{1em}{}
\titleformat{\subsubsection}{\normalfont\bfseries\color{mycolor}\centering}{\thesubsubsection}{1em}{}

%Commands

\usepackage[backend=bibtex, sorting=none]{biblatex}
\addbibresource{../bibliography.bib}


\usepackage{amsmath}
\usepackage{amsthm}
\usepackage{thmtools}
\usepackage{mathtools}
\usepackage{amsfonts}
\usepackage{dsfont}
\usepackage{yfonts}
\usepackage{amssymb}
\usepackage{cancel}
\usepackage{slashed}
\usepackage{feynmp-auto}


\newtheorem{principle}{Principle}[section]
\newtheorem{lemma}{Lemma}[section]
\theoremstyle{definition}
\newtheorem{example}{Example}[section]
\newtheorem{exercise}{Exercise}[section]
\renewcommand\qedsymbol{q.e.d.}

\let\oldproof\proof
\renewcommand{\proof}{\color{darkgray}\oldproof}

\theoremstyle{remark}
\newtheorem{case}{Case}

\newtheoremstyle{colored}{}{}{\itshape}{}{\color{mycolor}\normalfont\bfseries\indent}{}{\newline}{}

\declaretheorem[
  style=colored,
  name=Definition,
  numberwithin=section,
]{definition}

\declaretheorem[
  style=colored,
  name=Theorem,
  numberwithin=section,
]{theorem}

\declaretheorem[
  style=colored,
  name=Corollary,
  numberwithin=section,
]{corollary}

\declaretheorem[
  style=colored,
  name=Law,
  numberwithin=section,
]{law}

\declaretheorem[
  style=colored,
  name=Principle,
  numberwithin=section,
]{princ}

\newcommand{\dv}[2]{\frac{d#1}{d#2}}
\newcommand{\dvin}[3]{\frac{d#1}{d#2}\Big\vert_{#3}}
\newcommand{\dvd}[2]{\frac{d^2#1}{d#2^2}}
\newcommand{\dvf}[2]{\frac{\delta #1}{\delta #2}}
\newcommand{\pdv}[2]{\frac{\partial#1}{\partial#2}}
\newcommand{\pdvd}[3]{\frac{\partial^2 #1}{\partial#2 \partial#3}}
\newcommand{\pdvdu}[2]{\frac{\partial^2 #1}{\partial#2^2}}
\newcommand{\integ}[3]{\int_{#1}^{#2}d#3~}
\newcommand{\poi}[2]{[#1,~#2]}
\newcommand{\poiexp}[2]{\pdv{#1}{q^i} \pdv{#2}{p_i} - \pdv{#2}{q^i} \pdv{#1}{p_i}}

\newcommand{\comm}[2]{[#1,~#2]}
\newcommand{\set}[2]{\{#1\colon#2\}}
\newcommand{\inner}[2]{\langle#1,~#2\rangle}
\newcommand{\av}[1]{\langle#1\rangle}
\newcommand{\avp}[2]{\langle#1\rangle_{#2}}
\newcommand{\ket}[1]{\vert#1\rangle}
\newcommand{\bra}[1]{\langle#1\vert}
\newcommand{\braket}[2]{\langle#1\vert#2\rangle}  
   
\begin{document}  
 
\maketitle

\begin{abstract}
    In this note, we will study the mathematical devolopment to compute the S-matrix.  
\end{abstract}

\tableofcontents

\section{S-matrix}
 
    In this section, we will define the S-matrix and we will relate its elements to physical quantities, like cross sections and decay rates.

\subsection{Transition amplitudes}

    In quantum mechanics, experimentally measureable quantities are related to transition ampllitudes.
    \begin{definition}[Transition amplitude]
        Let $\ket{a}$ be a generic initial state and $\ket{b}$ a generic final state. Then, in the most generic case in which states are not normalised, the probability of the transition between the initial and the final state is given by 
        \begin{equation*}
            \mathcal P (a \rightarrow b) = \frac{|\braket{b}{a}|^2}{|\braket{b}{b}|^2 |\braket{a}{a}|^2} ~.
        \end{equation*}
    \end{definition}

    In Schroedinger picture, states depend on time while operators do not. 
    \begin{definition}[Transition amplitude in Schroedinger picture]
        Let $\ket{i, t_i}$ be a initial state at time $t_i$, $\ket{f, t_f}$ be a final state at time $t_f$. Then the probability of the transition between the initial and the final state is 
        \begin{equation*}
            \mathcal P (i, t_i \rightarrow f, t_f) = \frac{|\braket{f, t_f}{i, t_i}|^2}{|\braket{f, t_f}{f, t_f}|^2 |\braket{i, t_i}{i, t_i}|^2} ~.
        \end{equation*}
    \end{definition}

    In Heisenberg picture, states are time-independent while operators do not. Braket products in different pictures are related by 
    \begin{equation*}
        \braket{f, t_f}{i, t_i}_S = \bra{f} \hat S \ket{i}_H ~,
    \end{equation*}
    where $S$ is an operator that carries information about time evolution, called the S-matrix.
    \begin{definition}[Transition amplitude in Heisenberg picture]
        Let $\ket{i}$ be a initial state, $\ket{f}$ a final state, $\hat S$ the time evolution operator. Then the probability of the transition between the initial and the final state is 
        \begin{equation*}
            \mathcal P (i \rightarrow f) = \frac{|\bra{f} \hat S \ket{i}|^2}{|\braket{f}{f}|^2 |\braket{i}{i}|^2} ~.
        \end{equation*}
    \end{definition}

\subsection{Cross section}

    \begin{definition}[Cross section]
        Consider a scattering experiment. Let $N_{in}$ and $N_{out}$ be respectively the number of incoming and outgoing particles, $T$ the time of the experiment, $\Phi = N_{in} |\mathbf v| / V$ the flux of the incoming beam, where $V$ is the volume and $\mathbf v$ the velocity of the beam. Then the classical cross section is defined by 
        \begin{equation*}
            \sigma = \frac{N_{out}}{T \Phi} = \frac{V}{|\mathbf v| T} \frac{N_{out}}{N_{in}} ~.
        \end{equation*}
        Introducing the probability $\mathcal P = N_{out} / N_{in}$, its quantum mechanical counterpart is
        \begin{equation*}
            \sigma = \frac{V}{|\mathbf v| T} \mathcal P = \frac{N_{in}}{T \Phi} \mathcal P = \frac{1}{T \Phi} \mathcal P ~,
        \end{equation*}
        where we have redefined $\Phi = \Phi / N_{in}$ as the normalised one-particle flux. The differential cross section is 
        \begin{equation*}
            d \sigma = \frac{V}{|\mathbf v| T} d \mathcal P ~,
        \end{equation*}
        differential with respect to solid angle $d\Omega$ or energy $dE$. It has the dimension of an area, i.e. $[\sigma] = [L]^2$. 
    \end{definition}

\subsection{2 to n process}

    Consider a scattering experiment in which two incoming particle interact to produce $n$ outgoing particles
    \begin{equation*}
        p_1 + p_2 \rightarrow \{p_j\}_{j = 1}^n ~.
    \end{equation*}
    In perturbative theory, the S-matrix can be decomposed into 
    \begin{equation*}
        \hat S = \hat 1 + i \hat T ~,
    \end{equation*}
    where the identity $\hat 1$ represents no interactions, i.e. when $\ket{i} = \ket{f}$, and $\hat T$ describes deviations from it. Furthermore, since $4$-momentum is conserved, we can extract a delta from $\hat T$ to obtain
    \begin{equation*}
        i\hat T = (2\pi)^4 \delta^4 (p_1 + p_2 - {\textstyle \sum_j} p_j) i \hat{\mathcal M} ~,
    \end{equation*}
    where $\hat{\mathcal M}$ is the scattering amplitude.

    \begin{theorem}[Relation between cross section and S-matrix]
        In the approximation that interaction happens at finite time, the differential cross section of a $2 \rightarrow n$ process is 
        \begin{align*}
            d \sigma & = \frac{|\mathcal M|^2}{4 E_1 E_2 |\mathbf v_2 - \mathbf v_1|}  d\Pi_n \\ & = \frac{|\mathcal M|^2}{4 E_1 E_2 |\mathbf v_2 - \mathbf v_1|} \prod_j \frac{d^3 p_j}{(2\pi)^3 2 E_j} (2\pi)^4 \delta^4 (p_1 + p_2 - {\textstyle \sum_j} p_j)~.
        \end{align*}
    \end{theorem}
    \begin{proof}
        
    \end{proof}

\subsection{2 to 2 scattering}

    Consider the particular case in which there are two outgoing particles
    \begin{equation*}
        p_1 + p_2 \rightarrow p_3 + p_4 ~.
    \end{equation*}
    In the center of mass frame, the differential cross section is 
    \begin{equation*}
        d \sigma = \frac{1}{64 \pi^2 E_{cm}}^2 \frac{|\mathbf p_f|}{|\mathbf p_i|} |\mathcal M|^2 d\Omega ~,
    \end{equation*}
    where $|\mathbf p_i| = |\mathbf p_1| = |\mathbf p_2|$ and $|\mathbf p_f| = |\mathbf p_3| = |\mathbf p_4|$.
    \begin{proof}
        
    \end{proof}

    In the rest frame of particle $1$, the differential cross section is 
    \begin{equation*}
        d \sigma = \frac{1}{64 \pi^2 E_{cm}} \Big [ E_4 + E_3 \Big ( 1 - \frac{|\mathbf p_f|}{|\mathbf p_i|} \cos \theta \Big ) \Big ]^{-1} \frac{|\mathbf p_f|}{|\mathbf p_i|} |\mathcal M|^2 d\Omega ~.
    \end{equation*}
    \begin{proof}
        
    \end{proof}

\subsection{Decay rates}

    \begin{definition}[Decay rate]
        Consider a decay experiment. Let $\mathcal P$ be the probablity that a particle decays with mean lifetime $\tau$ and $T$ the time of the experiment. Then the decay rate is defined by 
        \begin{equation*}
            \Gamma = \frac{1}{\tau} = \frac{\mathcal P}{T} ~.
        \end{equation*}
        The differential decay rate is 
        \begin{equation*}
            d \Gamma = \frac{1}{T} d \mathcal P~,
        \end{equation*}
        differential with respect to solid angle $d\Omega$ or energy $dE$. It has the dimension of an inverse time, i.e. $[\Gamma] = [T]^{-1}$. 
    \end{definition}

\subsection{1 to n process}

    Consider a decay experiment in which a particle decays to produce $n$ outgoing particles
    \begin{equation*}
        p_1 \rightarrow \{p_j\}_{j = 1}^n ~.
    \end{equation*}

    \begin{theorem}[Relation between decay rate and S-matrix]
        In the approximation that interaction happens at finite time, the differential decay rate of a $1 \rightarrow n$ process is 
        \begin{equation*}
            d \Gamma = \frac{|\mathcal M|^2}{2 E_1}  d\Pi_n = \frac{|\mathcal M|^2}{2 E_1}  \prod_j \frac{d^3 p_j}{(2\pi)^3 2 E_j} (2\pi)^4 \delta^4 (p_1 - {\textstyle \sum_j} p_j)~.
        \end{equation*}
    \end{theorem}
    \begin{proof}
        
    \end{proof}

\section*{Propagators}

    In this section, we will relate S-matrix elements to time-ordered product of fields applied to interacting vacuum states.

\subsection{LSZ reduction formula} 

    \begin{theorem}[LSZ reduction formula]
        In the approximation that interaction happens at finite time, so that initial and final states are (asymptotic) free theory states, the S-matrix is given by
        \begin{align*}
            \bra{f} \hat S \ket{i} & = i \int dx_1 \exp(- i p_1 x_1) (\Box + m^2) \ldots i \int dx_1 \exp(i p_n x_n) (\Box + m^2) \\ & \quad \times \bra{\Omega} T \{ \phi(x_1) \ldots \phi(x_n)\} \ket{\Omega} ~,
        \end{align*}
        where $\ket{\Omega} \neq \ket{0}$ is the interacting vacuum, $-i$ in the exponent for initial states, $+i$ in the exponent for final states and $T$ is the time ordering operator which sorts all the operators in order to have time increasing from right to left. 
    \end{theorem}

    \begin{proof}
        
    \end{proof}


\subsection{Interaction picture}

    In Heisenberg picture, the dynamics is governed by the Hamiltonian $\hat H$. Fields evolve in time with the Heisenberg equation of motion 
    \begin{equation*}
        i \partial_t \hat \phi(t, \mathbf x) = [\hat \phi (t, \mathbf x), \hat H (t)] ~.
    \end{equation*}
    Its solution is 
    \begin{equation*}
        \hat \phi(t, \mathbf x) = \hat S^\dagger (t, t_0) \hat \phi(\mathbf x) \hat S(t, t_0) ~,
    \end{equation*}
    where $\hat S(t, t_0)$ is the time evolution operator that satisfies the Schroedinger equation 
    \begin{equation*}
        i \partial_t \hat S(t, t_0) = \hat H(t) \hat S(t, t_0) ~.
    \end{equation*}

    \begin{proof}

    \end{proof}

    Now, suppose that the Hamiltonian can be perturbatively decomposed into two pieces
    \begin{equation*}
        \hat H(t) = \hat H_0 + \hat  V(t) ~,
    \end{equation*}
    where $\hat H_0$ is exactly solved and $\hat V(t)$ is small. In interaction picture, operators evolve with $\hat H_0$, so that 
    \begin{equation*}
        \hat \phi_0(t, \mathbf x) = e^{i \hat H_0 (t - t_0)} \hat \phi(\mathbf x) e^{-i \hat H_0 (t - t_0)}
    \end{equation*}
    where $t_0$ is a time in which Schroedinger and Heisenberg picture field coincide. Therefore 
    \begin{equation*}
        \phi(t, \mathbf x)
    \end{equation*}

\subsection{Vacuum matrix elements}

    \begin{theorem}[Relation between interacting and free vacuum matrix elements]
        \begin{align*}
            \bra{\Omega} T \{\phi(x_1) \ldots \phi(x_n)\} \ket{\Omega} & = \frac{\bra{0} T \{\phi_0(x_1) \ldots \phi_0(x_n) \exp(- i \int_{-\infty}^\infty dt ~ V_I(t) )\} \ket{0}}{\bra{0} T \{\exp(- i \int_{-\infty}^\infty dt ~ V_I(t) ) \} \ket{0}} \\ & = \frac{\bra{0} T \{\phi_0(x_1) \ldots \phi_0(x_n) \exp(i \int d^4 x ~ \mathcal L_{int} [\phi_0] )\} \ket{0}}{\bra{0} T \{\exp(i \int d^4 x ~ \mathcal L_{int} [\phi_0] ) \} \ket{0}} ~.
        \end{align*}
    \end{theorem}

    \begin{proof}
        
    \end{proof}

\section{Wick's theorem}

\section*{Back}

\nocite{qftlecture} 
\nocite{schwartz}  
\printbibliography

\immediate\write18{mv \jobname.pdf ../../../pdf/\jobname.pdf}

\end{document} 

 