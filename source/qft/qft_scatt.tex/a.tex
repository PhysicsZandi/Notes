\documentclass[a4paper]{article} 

\def\ncourse {Quantum Field Theory}
\def\ntopic {3 - scattering processes}
 
\makeatletter

%Title
\def\nauthor{Matteo Zandi}

\title{\huge \ncourse \\ \Large \ntopic}

\author{\color{mycolor}\nauthor}
\date{\today}

%Size
\usepackage{geometry}
%\geometry{a4paper, top = 4cm, bottom = 4cm, left = 3cm, right = 4cm}
\geometry{
    papersize={379pt, 699pt},
    textwidth=345pt,
    textheight=596pt,
    left=17pt,
    top=54pt,
    right=17pt
  }

%Package
\usepackage{lipsum}
\usepackage{xcolor} \xdefinecolor{mycolor}{RGB}{0,175,179} 
\usepackage{hyperref} \hypersetup{colorlinks, linkcolor={mycolor}, citecolor={mycolor}, urlcolor={mycolor}}
\usepackage{titlesec}
\titleformat{\section}{\newpage\normalfont\Large\bfseries\color{mycolor}\centering}{\thesection}{1em}{}
\titleformat{\subsection}{\normalfont\large\bfseries\color{mycolor}\centering}{\thesubsection}{1em}{}
\titleformat{\subsubsection}{\normalfont\bfseries\color{mycolor}\centering}{\thesubsubsection}{1em}{}

%Commands

\usepackage[backend=bibtex, sorting=none]{biblatex}
\addbibresource{../bibliography.bib}


\usepackage{amsmath}
\usepackage{amsthm}
\usepackage{thmtools}
\usepackage{mathtools}
\usepackage{amsfonts}
\usepackage{dsfont}
\usepackage{yfonts}
\usepackage{amssymb}
\usepackage{cancel}
\usepackage{slashed}
\usepackage{feynmp-auto}


\newtheorem{principle}{Principle}[section]
\newtheorem{lemma}{Lemma}[section]
\theoremstyle{definition}
\newtheorem{example}{Example}[section]
\newtheorem{exercise}{Exercise}[section]
\renewcommand\qedsymbol{q.e.d.}

\let\oldproof\proof
\renewcommand{\proof}{\color{darkgray}\oldproof}

\theoremstyle{remark}
\newtheorem{case}{Case}

\newtheoremstyle{colored}{}{}{\itshape}{}{\color{mycolor}\normalfont\bfseries\indent}{}{\newline}{}

\declaretheorem[
  style=colored,
  name=Definition,
  numberwithin=section,
]{definition}

\declaretheorem[
  style=colored,
  name=Theorem,
  numberwithin=section,
]{theorem}

\declaretheorem[
  style=colored,
  name=Corollary,
  numberwithin=section,
]{corollary}

\declaretheorem[
  style=colored,
  name=Law,
  numberwithin=section,
]{law}

\declaretheorem[
  style=colored,
  name=Principle,
  numberwithin=section,
]{princ}

\newcommand{\dv}[2]{\frac{d#1}{d#2}}
\newcommand{\dvin}[3]{\frac{d#1}{d#2}\Big\vert_{#3}}
\newcommand{\dvd}[2]{\frac{d^2#1}{d#2^2}}
\newcommand{\dvf}[2]{\frac{\delta #1}{\delta #2}}
\newcommand{\pdv}[2]{\frac{\partial#1}{\partial#2}}
\newcommand{\pdvd}[3]{\frac{\partial^2 #1}{\partial#2 \partial#3}}
\newcommand{\pdvdu}[2]{\frac{\partial^2 #1}{\partial#2^2}}
\newcommand{\integ}[3]{\int_{#1}^{#2}d#3~}
\newcommand{\poi}[2]{[#1,~#2]}
\newcommand{\poiexp}[2]{\pdv{#1}{q^i} \pdv{#2}{p_i} - \pdv{#2}{q^i} \pdv{#1}{p_i}}

\newcommand{\comm}[2]{[#1,~#2]}
\newcommand{\set}[2]{\{#1\colon#2\}}
\newcommand{\inner}[2]{\langle#1,~#2\rangle}
\newcommand{\av}[1]{\langle#1\rangle}
\newcommand{\avp}[2]{\langle#1\rangle_{#2}}
\newcommand{\ket}[1]{\vert#1\rangle}
\newcommand{\bra}[1]{\langle#1\vert}
\newcommand{\braket}[2]{\langle#1\vert#2\rangle}  
   
\begin{document}  
 
\maketitle

\begin{abstract}
    In this note, we will study all the important scattering processes for $\phi^3$, sQED, Yukawa and QED.
\end{abstract}

\tableofcontents
\newpage 

\section{Scalar}

\subsection{2 to 2}

    Consider for $\phi^3$ theory the scattering \[\phi \phi \rightarrow \phi \phi\] at tree level. There are three possible Feynman's diagram:
    \begin{figure}[ht!]
        \centering
        \begin{fmffile}{scal_1} 
            \begin{fmfgraph*}(100, 100)  
                \fmfleft{i1,i2}
                \fmfright{o1,o2}
                \fmf{scalar,label=$p_1$}{i1,w1} 
                \fmf{scalar,label=$p_2$}{i2,w1} 
                \fmf{scalar,label=$p_1 + p_2$}{w1,w2} 
                \fmf{scalar,label=$p_3$}{w2,o1}
                \fmf{scalar,label=$p_4$}{w2,o2}
            \end{fmfgraph*}
        \end{fmffile} 
    \end{figure}  
    \newline which gives 
    \begin{equation*}
        i \mathcal M = (- i \lambda) \frac{i}{(p_1 + p_2)^2 - m^2} (- i \lambda) = - \frac{i \lambda^2 }{s - m^2} ~.
    \end{equation*}
    \begin{figure}[ht!]
        \centering
        \begin{fmffile}{scal_2} 
            \begin{fmfgraph*}(100, 100)  
                \fmfleft{i1,i2}
                \fmfright{o1,o2}
                \fmf{scalar,label=$p_2$}{i1,w1} 
                \fmf{scalar,label=$p_1$}{i2,w2} 
                \fmf{scalar,label=$p_1 - p_3$}{w2,w1}  
                \fmf{scalar,label=$p_4$}{w1,o1}
                \fmf{scalar,label=$p_3$}{w2,o2} 
            \end{fmfgraph*} 
        \end{fmffile} 
    \end{figure} 
    \newline which gives 
    \begin{equation*}
        i \mathcal M = (- i \lambda) \frac{i}{(p_1 - p_3)^2 - m^2} (- i \lambda) = - \frac{i \lambda^2 }{t - m^2} ~.
    \end{equation*}
    \begin{figure}[ht!]
        \centering
        \begin{fmffile}{scal_3} 
            \begin{fmfgraph*}(100, 100)  
                \fmfleft{i1,i2}
                \fmfright{o1,o2}
                \fmf{scalar, label=$p_2$}{i1,v1}  
                \fmf{phantom}{v1,o1}
                \fmf{scalar, label=$p_1$}{i2,v2} 
                \fmf{phantom}{v2,o2}
                \fmf{scalar, label=$p_1 - p_4$}{v1,v2} 
                \fmf{scalar,tension=0,label=$p_3$}{v1,o2}
                \fmf{scalar,tension=0,label=$p_4$}{v2,o1} 
            \end{fmfgraph*} 
        \end{fmffile} 
    \end{figure} 
    \newline which gives 
    \begin{equation*}
        i \mathcal M = (- i \lambda) \frac{i}{(p_1 - p_4)^2 - m^2 + i \epsilon} (- i \lambda) = - \frac{i \lambda^2}{u - m^2} ~.
    \end{equation*}
    Putting everything together, we find
    \begin{equation*}
        i \mathcal M = - i \lambda^2 \Big ( \frac{1}{s - m^2} + \frac{1}{t - m^2} + \frac{1}{u- m^2} \Big) ~,
    \end{equation*}
    hence
    \begin{equation*}
        |\mathcal M|^2 = \lambda^4 \Big ( \frac{1}{s - m^2} + \frac{1}{t - m^2} + \frac{1}{u - m^2} \Big)^2 ~.
    \end{equation*}
    For example, the differential cross section in the center of mass is
    \begin{equation*}
        \dv{\sigma}{\Omega} = \frac{\lambda^4}{64 \pi^2 s} \Big ( \frac{1}{s - m^2} + \frac{1}{t - m^2} + \frac{1}{u - m^2} \Big)^2 ~. 
    \end{equation*}

\section{sQED} 

\subsection{Moller scattering}

    Consider for sQED theory the scattering \[\phi \phi \rightarrow \phi \phi\] at tree level. There are two possible Feynman's diagram:
    \begin{figure}[ht!]
        \centering
        \begin{fmffile}{sqed_1} 
            \begin{fmfgraph*}(100, 100)  
                \fmfleft{i1,i2}
                \fmfright{o1,o2}
                \fmf{scalar,label=$p_2$}{i1,w1} 
                \fmf{scalar,label=$p_1$}{i2,w2} 
                \fmf{photon,label=$p_1 - p_3$}{w2,w1}   
                \fmf{scalar,label=$p_4$}{w1,o1}
                \fmf{scalar,label=$p_3$}{w2,o2} 
            \end{fmfgraph*} 
        \end{fmffile} 
    \end{figure} 
    \newline which gives, using Mandelstam's variables properties in \eqref{mand},
    \begin{align*}
        i \mathcal M & = (- i e) (p_1^\mu + p_3^\mu) \frac{-i g_{\mu\nu}}{(p_1 - p_3)^2} (- i e) (p_2^\nu + p_4^\nu) = - \frac{i e^2}{t} (p_1+ p_3) (p_2 + p_4) \\ & = - \frac{i e^2}{t} ( p_1 p_2+ p_1 p_4 + p_3 p_2 + p_3 p_4 ) \\ & = - \frac{i e^2}{t} \Big ( \frac{s - m^2_1 - m_2^2}{2} - \frac{u - m_1^2 - m_4^2}{2} - \frac{u - m_2^2 - m_3^2}{2} + \frac{s - m_3^2 - m_4^2}{2} \Big ) \\ & = - \frac{i e^2}{t} \Big ( \frac{s}{2} - \frac{u}{2} - \frac{u}{2} + \frac{s}{2} \Big ) = - \frac{i e^2}{2t} (s - u) ~.
    \end{align*}
    \begin{figure}[ht!]
        \centering
        \begin{fmffile}{sqed_2} 
            \begin{fmfgraph*}(100, 100)  
                \fmfleft{i1,i2}
                \fmfright{o1,o2}
                \fmf{scalar, label=$p_2$}{i1,v1}  
                \fmf{phantom}{v1,o1}
                \fmf{scalar, label=$p_1$}{i2,v2} 
                \fmf{phantom}{v2,o2}
                \fmf{photon, label=$p_1 - p_4$}{v1,v2}  
                \fmf{scalar,tension=0,label=$p_3$}{v1,o2}
                \fmf{scalar,tension=0,label=$p_4$}{v2,o1} 
            \end{fmfgraph*} 
        \end{fmffile} 
    \end{figure} 
    \newline which gives, using Mandelstam's variables properties in \eqref{mand},
    \begin{align*}
        i \mathcal M & = (- i e) (p_1^\mu + p_4^\mu) \frac{-i g_{\mu\nu}}{(p_1 - p_4)^2} (- i e) (p_2^\nu + p_3^\nu) = - \frac{i e^2}{u} (p_1+ p_4) (p_2 + p_3) \\ & = - \frac{i e^2}{u} ( p_1 p_2 + p_1 p_3 + p_4 p_2 + p_4 p_3 ) \\ & = - \frac{i e^2}{u} \Big ( \frac{s - m^2_1 - m_2^2}{2} - \frac{t - m_1^2 - m_3^2}{2} - \frac{t - m_2^2 - m_4^2}{2} + \frac{s - m_3^2 - m_4^2}{2} \Big ) \\ & = - \frac{i e^2}{u} \Big ( \frac{s}{2} - \frac{t}{2} - \frac{t}{2} + \frac{s}{2} \Big ) = - \frac{i e^2}{2u} (s - t) ~.
    \end{align*}
    Putting everything together, we find
    \begin{equation*}
        i \mathcal M = - \frac{i e^2}{2} \Big ( \frac{s - u}{t} + \frac{s- t}{u} \Big ) ~,
    \end{equation*}
    hence
    \begin{equation*}
        |\mathcal M|^2 = \frac{e^4}{4} \Big ( \frac{s - u}{t} + \frac{s- t}{u} \Big )^2 ~.
    \end{equation*}
    For example, the differential cross section in the center of mass is
    \begin{equation*}
        \dv{\sigma}{\Omega} = \frac{e^4}{256 \pi^2} \Big ( \frac{s - u}{t} + \frac{s- t}{u} \Big )^2 ~. 
    \end{equation*}

\subsection{Compton scattering}

    Consider for sQED theory the scattering \[\gamma \phi \rightarrow \gamma \phi\] at tree level. There are three possible Feynman's diagram:
    \begin{figure}[ht!]
        \centering
        \begin{fmffile}{sqed_3} 
            \begin{fmfgraph*}(100, 100)  
                \fmfleft{i1,i2}
                \fmfright{o1,o2}
                \fmf{photon,label=$p_2$}{i1,w1} 
                \fmf{scalar,label=$p_1$}{i2,w1} 
                \fmf{scalar,label=$k$}{w1,w2} 
                \fmf{photon,label=$p_4$}{w2,o1}
                \fmf{scalar,label=$p_3$}{w2,o2}  
            \end{fmfgraph*} 
        \end{fmffile} 
    \end{figure} 
    \newline which gives
    \begin{align*}
        i \mathcal M & = (-i e) \epsilon_{2 \mu} (p_1^\mu + p_1^\mu + p_2^\mu) \frac{i}{s - m^2} (-i e) (p_3^\nu + p_3^\nu + p_4^\nu) \epsilon_{4 \nu}^* \\ & = - \frac{ i e^2}{s - m^2} \epsilon_{2 \mu} (2 p_1^\mu + p_2^\mu) (2 p_3^\nu + p_4^\nu) \epsilon_{4 \nu}^* ~.
    \end{align*}
    \begin{figure}[ht!]
        \centering
        \begin{fmffile}{sqed_4} 
            \begin{fmfgraph*}(100, 100)
                \fmfleft{i1,i2}
                \fmfright{o1,o2} 
                \fmf{photon,label=$p_2$}{i1,w1} 
                \fmf{scalar,label=$p_1$}{i2,w2}
                \fmf{scalar}{w2,w1}  
                \fmf{scalar,label=$p_4$}{w1,o1} 
                \fmf{photon,label=$p_3$}{w2,o2}   
            \end{fmfgraph*} 
        \end{fmffile} 
    \end{figure} 
    \newline which gives
    \begin{align*}
        i \mathcal M & = (- i e) \epsilon_{2 \mu} (p_1^\mu + p_1^\mu - p_3^\mu) \frac{i}{t - m^2} (-i e) (p_4^\mu + p_4^\mu - p_2^\mu) \epsilon_{4 \nu}^* \\ & = - \frac{ie^2}{t - m^2} \epsilon_{2\mu} (2 p_1^\mu - p_3^\mu) (2 p_4^\mu - p_2^\mu) \epsilon_{4\mu}^* ~.
    \end{align*}
    \begin{figure}[ht!]
        \centering
        \begin{fmffile}{sqed_5} 
            \begin{fmfgraph*}(100, 100)  
                \fmfleft{i1,i2}
                \fmfright{o1,o2}
                \fmf{photon,label=$p_2$}{i1,v1}  
                \fmf{scalar,label=$p_1$}{i2,v1}
                \fmf{scalar,label=$p_3$}{v1,o2}
                \fmf{photon,label=$p_4$}{v1,o1}
            \end{fmfgraph*} 
        \end{fmffile} 
    \end{figure} 
    \newline which gives
    \begin{align*}
        i \mathcal M & = \epsilon_{2 \mu} 2 i e^2 g_{\mu\nu} \epsilon_{4 nu}^* ~.
    \end{align*}
    Putting everything together, we find
    \begin{align*} 
        i \mathcal M & = - \frac{ i e^2}{s - m^2} \epsilon_{2 \mu} (2 p_1^\mu + p_2^\mu) (2 p_3^\nu + p_4^\nu) \epsilon_{4 \nu}^* \\ & \quad - \frac{ie^2}{t - m^2} \epsilon_{2\mu} (2 p_1^\mu - p_3^\mu) (2 p_4^\mu - p_2^\mu) \epsilon_{4\mu}^* + \epsilon_{2 \mu} 2 i e^2 g^{\mu\nu} \epsilon_{4 nu}^* ~.
    \end{align*}
    Now, recall that by Lorentz gauge $\epsilon^{\mu i} p_{\mu i}$ for $i = 1,2,3,4$. Furthermore, 
    hence
    \begin{equation*}
        |\mathcal M|^2 = \frac{e^4}{4} \Big ( \frac{s - u}{t} + \frac{s- t}{u} \Big )^2 ~.
    \end{equation*}
    For example, the differential cross section in the center of mass is
    \begin{equation*}
        \dv{\sigma}{\Omega} = \frac{e^4}{256 \pi^2} \Big ( \frac{s - u}{t} + \frac{s- t}{u} \Big )^2 ~. 
    \end{equation*}


\section{Yukawa}

\section{QED}



\appendix 

\section{Useful identities}

    \begin{equation}\label{mand}
    \begin{aligned}
        & p_1 p_2 = \frac{s - m_1^2 - m_2^2}{2} = p_3 p_4 = \frac{s - m_3^2 - m_4^2}{2} ~, \\ & p_1 p_3 = - \frac{t - m_1^2 - m_3^2}{2} = p_2 p_4 = - \frac{t - m_2^2 - m_4^2}{2} ~, \\ &  p_1 p_4 = - \frac{u - m_1^2 - m_4^2}{2} = p_2 p_3 = - \frac{u - m_2^2 - m_3^2}{2} ~.
    \end{aligned} 
    \end{equation}

\section*{Back}



\nocite{qftlecture} 
\nocite{schwartz}  
\printbibliography

\immediate\write18{mv \jobname.pdf ../../../pdf/\jobname.pdf}

\end{document} 

 